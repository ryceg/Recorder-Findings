In the context of my thesis, `Writing Wrong Right: Composing With Extended Techniques', I examine the ways in which we can remove the barriers that composers and performers face with extended techniques.
As part of the thesis, I explore the semiology of notation, and deconstruct it to establish a `cookbook' of elements, which can be used to construct new notation for techniques that adheres to the unwritten `style guide' of existing Western musical notation.
This paper is a practical example of this; in it, I use autoethnography to show my experiences constructing both a technique, and a type of notation. 
This is intended to be representative, as a technique maps an action to a symbol, while a form of notation maps a mode of interacting with an instrument to the score;
it is intended that the reader can extrapolate from my findings along with the cookbook to construct notation to suit their purposes as required.

In my first year of composition, I wrote an ambitious work called `Music for String Instrument and Audience', in which the audience received a full score of the work along with the programme for the recital.
The work was notated as a colour-coded flowchart, and it `talked the audience through' the work; elements of chance were included for both the performer and audience to act upon.
These included instructions such as `play until you get bored, then move to the next box', `play until you hear a rustling of paper in the audience', and `if you are wearing plain socks, play this instead'.
My intent was to both demonstrate an understanding of the hegemonic hierarchy of the composer-performer-audience relationship, and challenge it, as usage of extended techniques was missing from my high school portfolio.
The performance was marred by the lights not going on, resulting in the audience not being able to read the scores in the dimly lit room, but despite the catastrophic premiere, the work provoked an interest in me regarding the systems that were at play, and I attribute it at least partially with my fascination with extended techniques.
