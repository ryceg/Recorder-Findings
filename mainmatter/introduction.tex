In the context of my thesis, `Writing Wrong Right: Composing With Extended Techniques', I examine the ways in which we can remove the barriers that composers and performers face with extended techniques.
As part of the thesis, I explore the semiology of notation, and deconstruct it to establish a `cookbook' of elements, which can be used to construct new notation for techniques that adheres to the unwritten `style guide' of existing Western musical notation.
This paper is a practical example of this; in it, I use autoethnography to show my experiences constructing both a technique, and a type of notation. 
This is intended to be representative, as a technique maps an action to a symbol, while a form of notation maps a mode of interacting with an instrument to the score;
it is intended that the reader can extrapolate from my findings along with the cookbook to construct notation to suit their purposes as required.
