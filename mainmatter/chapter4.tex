
\section{Frank Zappa, Kate Soper, and Pepe Silvia}
David Dockery is a drummer that went viral years ago with his unique style of drumming in time to speech patterns of TV shows. 
The effect is not dissimilar to that of Frank Zappa's song ``The Dangerous Kitchen'' amongst others, from his album ``The Man from Utopia''. 
This effect of speech being done in time with the music can also be seen in Kate Soper's work ``Only the Words Themselves Mean What They Say''. 
Transcribing speech patterns and quantizing it to conform to a metrical grid is well established as a technique of introducing off-kilter rhythms that feel natural, however as Dockery's drumming shows, the works that are most in line with regular emotive speech contain multiple tempo and meter changes- 
showing that speech patterns are not typical iambic pentameter.

These transcriptions give us an insight into the mind of the composers, and I posit that the notated rhythms are notated thus not because the rhythms are important, but because the Western notation system does not give us adequate tools to notate speech-like cadences. 
In order to communicate the intended speech-like effect, the composers notate using the tools available, fitting their isochronous rhythms to the grid of Western notation.
This relies on either an explicit statement that the rhythms are unimportant, or the performer interpreting the subtext of the work correctly and surmising that without explicit direction.
This transcription process is lengthy, though, and when the intent is to mimic a relaxed and conversational style of speech, and not to reproduce those exact rhythms as notated, there is a degree of specificity of rhythm that is unnecessary, needlessly complicating the work.

My suggested notation is similar to non-barred music, as found in the works of Berio's Sequenzas.
It involves the removal of barlines, and stems of notes. 
Rhythms are derived from the lyric line underneath the stave. 

One of the issues with this is that it assumes both a level of flexibility in the rhythms, and a level of uniformity in how performers will interpret it; if the performer differs dramatically from the composer's rough intended idea, the work may fall out of `sync' with itself.
Mitigating this is the second version of notation, in which the beams are slightly wavy, contouring in the same manner as the \\approx symbol. 
This reference to a pre-existing symbol for the notation of `approximately' is intended to aid comprehension by not reinventing the wheel.
This wavy-beamed notation can be used in existing Western style barred music, as well as convey a rough sense of intended rhythm when it differs from how a text would be spoken out loud.
This is useful because it gives the composer a way to `override' the performer's natural diction, without retreating back to the traditional Western notation system, which as discussed, is sometimes not suited for the notation of unfixed rhythms.

Composers that wish for a slightly more granular level of control of the rhythms can use the modified staving as notated 

One of the benefits of this notation system is that it maintains compatibility with the traditional Western notation system, and can be 

The simplest and most evident example of this technique's potential is when it is used with an instrument that can be spoken into, such as the recorder. 
In the example work provided, the text is legible, but also can be used in tandem with regular notation systems, in a manner similar to that of Lutoslawski's stochastic aleatoricism. 
Like Lutoslawski's system, there is 
