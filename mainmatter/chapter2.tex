\section{Methodology}


\section{Case Study}

The composer and recorder player Claire Farrell approached me, asking for my opinion on how to notate an extended technique that she had been developing. 
The technique involves the recorder player covering the window hole with their index finger or hand, which results in the recorder producing a whistle like effect. 
Air pressure determines the pitch, and the degree to which the window hole is covered determines the fundamental's presence or lack thereof, with full occlusion of the window producing solely the harmonic. 

This is an ideal place in which to explore the ways in which we map actions into notation; the question becomes one of how this can be achieved in the most recognisable and orderly manner. 
Our goal of establishing a defined notation system is to ensure that it is as clear and easy to understand as possible. 

Thus began an informative exploration through the various ways that we can map sound onto an instrument. 
This is an especially interesting topic to me, as I am interested in the ways in which the semiotics of notation influence its parsing. 
We can treat this development of the notation of a new technique as a case study for best practices, and establish some ground rules on how the written form of Western art music can be adapted to accomodate new techniques that fall outside of the initial set.
